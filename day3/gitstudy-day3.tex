\documentclass
[printout]
% [handout]
{beamer}

\usepackage[utf8]{inputenc}
\usepackage{graphics}
\usepackage{newcent}
\usepackage[absolute,overlay]{textpos}
\usepackage{calc}
\usetheme{Goettingen}
\usecolortheme{beaver}
\title[Git Study Day 3\\Application]{Application of Git}
\subtitle{CAT\&DOG Git study 2013\\Day 3}
\author{Eon Jeong}
\date{\today}

\begin{document}

\begin{frame}
  \titlepage
\end{frame}

\setcounter{section}{-1}

\section{Contents}

\begin{frame}{Contents}
  \tableofcontents
\end{frame}

\setcounter{section}{0}

\section{Applying patches and forks}

\begin{frame}
  \sectionpage
\end{frame}

\begin{frame}{How to compare?}
  \fontsize{7pt}{9pt}\selectfont
  \texttt{
    \begin{tabular}{|r|l}
      1& \#include \textless stdio.h\textgreater \\
      2& \\
      3& int main() \textbraceleft\\
      4& \ \ \ \ const char format[11] = "Hello \%s!\textbackslash n";\\
      5& \ \ \ \ char buf[80] = "World";\\
      6& \ \ \ \ printf(format, buf);\\
      7& \ \ \ \ return 0;\\
      8& \textbraceright\\
      9& \\
    \end{tabular}
    \begin{tabular}{|r|l}
      1& \#include \textless stdio.h\textgreater \\
      2& \#include \textless time.h\textgreater \\
      3& \\
      4& int main() \textbraceleft\\
      5& \ \ \ \ const char format[11] = "Hello \%s!\textbackslash n";\\
      6& \ \ \ \ char buf[80];\\
      7& \ \ \ \ time\_t now = time(NULL);\\
      8& \ \ \ \ strftime(buf, 20, "\%F \%T", localtime(\&now));\\
      9& \ \ \ \ printf(format, buf);\\
      10& \ \ \ \ return 0;\\
      11& \textbraceright\\
      12& \\
    \end{tabular}
  }
\end{frame}

\begin{frame}{Diff}
  \fontsize{7pt}{9pt}\selectfont
  \texttt{
    \begin{tabular}{|r|r|l}
      1& 1& \#include \textless stdio.h\textgreater \\
      & 2& \#include \textless time.h\textgreater \\
      2& 3& \\
      3& 4& int main() \textbraceleft\\
      4& 5& \ \ \ \ const char format[11] = "Hello \%s!\textbackslash n";\\
      5&  & \ \ \ \ char buf[80] = "World";\\
      & 6& \ \ \ \ char buf[80];\\
      & 7& \ \ \ \ time\_t now = time(NULL);\\
      & 8& \ \ \ \ strftime(buf, 20, "\%F \%T", localtime(\&now));\\
      6& 9& \ \ \ \ printf(format, buf);\\
      7& 10& \ \ \ \ return 0;\\
      9& 11& \textbraceright\\
      9& 12& \\
    \end{tabular}
  }
\end{frame}

\begin{frame}{\texttt{diff}}
  Show line-by-line difference (mininum edit) between commits or workspace.\pause\\
  \begin{itemize}
  \item Compare file(s) between commits. (2 or \textit{more})
  \item - and +, not \textless and \textgreater.
  \item Display revision ID and adjacent common lines.
  \end{itemize}\pause
  We can use \texttt{format-patch} to generate patch file from diff.
\end{frame}

\begin{frame}{\texttt{remote}}
  Manage remote tracked repositories.
  \begin{itemize}
  \item \texttt{remote add}\\\texttt{remote rename}\\\texttt{remote remove}
  \item \texttt{remote set-head}\\\texttt{remote set-branches}\\\texttt{remote set-url}
  \item \texttt{remote prune}\\\texttt{remote update}
  \end{itemize}
\end{frame}

\begin{frame}{\texttt{merge}}
  Join two or more development histories together.
  \begin{itemize}
  \item Use history from their past common revision, if possible.
  \end{itemize}
\end{frame}

\begin{frame}{\texttt{fetch}}
  Think about pull without merge. Just update references and objects.
\end{frame}

\section{Maintaining a repository}

\begin{frame}
  \sectionpage
\end{frame}

\begin{frame}{\texttt{branch}}
  List, create, or delete codelines.
\end{frame}

\begin{frame}{\texttt{tag}}
  List, create, delete or verify a tag object, GPG-signed.
\end{frame}

\begin{frame}{\texttt{rebase}}
  Forward commits to updated stream baseline by differences.
\end{frame}

\begin{frame}{\texttt{revert}}
  Revert some commits without stripping them.
\end{frame}

\section{Other features}
\begin{frame}
  \sectionpage
\end{frame}

\begin{frame}{\texttt{reset}}
  Transit HEAD to the specified state.
\end{frame}

\begin{frame}{\texttt{stash}}
  Temporarily store current changes and clean workspace.
\end{frame}

\begin{frame}{\texttt{config}}
  Handle options, for repository or global(user-wide).
\end{frame}

\begin{frame}{\texttt{.gitignore}}
  We can specify untracked files by name specs, e.g. for temporary files.
  \begin{itemize}
  \item Wildcard(*) can be used.
  \item Without \texttt{/}, pplied recursively.
  \item Files already staged are not affected.
  \end{itemize}
\end{frame}

\begin{frame}{\texttt{.gitmodules}}
  Some workspace is required to contain other repository \textit{module} inside, e.g. for library use.\\
  cf. \texttt{submodule}
\end{frame}

\section{Thanks}

\begin{frame}{Thank you.}
  \begin{enumerate}
  \item CAT
  \item DOG\pause
  \item ???
  \item PROFIT!
  \end{enumerate}
\end{frame}

\end{document}
